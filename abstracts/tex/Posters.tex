\poster{An unsupervised Machine Learning technique for dimensionality reduction of single cell data}
{Helena Andres-Terre} {University of Cambridge}
The introduction of single cell RNA-seq data was a major breakthrough in the field of biology, and particularly useful in research areas such as comparative transcriptomics or disease studies. It allows the characterization of gene expression levels for individual cells, and the potential to observe different stages of the Stem cell differentiation process.
These datasets are known to be sparse and highly dimensional, with a large number of genes describing each cell. In order to interpret the experimental results, one of the main objectives is to identify the most relevant features of the underlying processes. After a first cut on the number of accounted genes based on their variability, current techniques use dimensionality reduction methods such as Principal Component Analysis (linear) or tSNE (non-linear). The new components are then used for plotting, performing further analysis on classification tasks or to describe differentiation processes. While these methods have been proved valid to address the aforementioned challenges, they also present some restrictions when trying to characterize the middle states of differentiation.
We present an unsupervised Machine Learning technique for dimensionality reduction of single cell data. We used Variational Auto-Encoders to extract a number of significant components that characterize individual cells based on their gene expression, using a deep learning bottle-neck approach.
The variational nature of this technique allows for certain levels of stochasticity in the original data, while learning an encoded representation that can be used to reconstruct and generate new samples.
The methods that are based on linear dimensionality reduction techniques often require strong assumptions and constraints; for instance locally homogeneous distribution of samples in the high dimensional space, or not accounting for low variance dimensions. Variational Auto-Encoders (VAEs) provide an en- coding capable to identify and separate among relevant features in the dataset.
VAEs can also be used to identify sets of relevant genes or drivers of the different processes captured by the latent encoding. Their generative potential also allows the exploration of the latent space, which can lead to defining a theoretical energy landscape that describes trajectories of differentiation.

\poster{SusNet: a global retinal co-expression network}
{Annamaria Carissimo} {Istituto per le Applicazioni del Calcolo M.Picone, Naples, CNR}
Network analysis provides a useful framework to visualize and analyze complex biological problems. In biological networks, transcripts, genes or proteins are represented as nodes, and relationships between them as edges. These interactions can be reconstructed by inference methods starting from expression profiles. Co-expression networks use the transcriptional concordance of two gene expression profiles to build undirected graph representations of the biological system under observation. In our study, we generated a co-expression network of the adult porcine retina, SusNet,  by calculating pairwise gene correlation among 47 Sus Scrofa Large White retina samples sequenced by RNA-Seq.  We showed that SusNet captures the pan-retinal regulatory structure associated to retina-specific TFs. We further showed that mapping differentially expressed genes (DEGs) following somatic repression of the rod-specific gene Rhodopsin on SusNet enables the identification of a sub-GRN operating in a subcellular compartment.


\poster{COSMONET}{Antonella Iuliano}{Telethon Institute of Genetics and Medicine (TIGEM), Pozzuoli, Italy}
We present an R package called COSMONET where screening-network methods are implemented for the prediction of survival outcome in cancer patients. The novelty is the combination of different types of screenings (biomedical-driven, data-driven, the union of both) and network-regularized Cox methods. This approach allows to improve the prediction capabilities, to discriminate patients in high-and low-risk groups using few potential biomarkers, and to help clinicians in the management of patients.

\poster{Graphical Markov Models with an Application on Traffic Accident Data}{Gamze OZEL KADILAR }{ Hacettepe University, Ankara, Turkey}
Undirected graphical models are widely used for modeling, visualization, inference, and exploratory analysis of multivariate data with wide-ranging applications. Graphical models are models based on graphs in which nodes represent random variables, and the edges represent conditional independence assumptions. Hence they provide a compact representation of joint probability distributions. Graphical Markov models started to be developed after 1970 as special subclasses of log-linear models for contingency tables and of joint Gaussian distributions, where conditional independence constraints are imposed such that conditioning is on all the other variables. The study of these models is an active research area, with many questions still open. In this study, graphical Markov models are described. Then, interpretations are illustrated with an application based on traffic accident data of Turkey. Furthermore, some of the more recent, important results for sequences of regressions are summarized.
Keywords: graphical models; Markov property; categorical data analysis, traffic accident data.

\poster{Identification of cell subpopulations using ensMAP-DP approach}{Monika Krzak}{ Istituto per le Applicazioni del Calcolo M.Picone, Naples, CNR}
Single-cell RNA sequencing (scRNAseq) has emerged as an important technology that allows profiling gene expression at single-cell resolution. The great potential of this technique lies in the possibility to infer cellular diversity within the same organ, tissue or group of cells of interest. In the last years, several studies have been carried out for identifying novel or known cell populations. Despite the great collection of available methods, an accurate detection of cell subpopulations remains unresolved and several issues are still open. For example, most of the algorithms require to provide a fixed number of
desired subpopulations. This might be a drawback when no prior knowledge about cell population is available or when the aim is to identify novel subtypes of cells. Motivated by these reasons, we developed a new method, ensMAP-DP, that uses probabilistic mixture modeling to reveal latent cell subpopulations. The method consists of several steps: first, it selects a number of most relevant features (i.e. so-called highly variable genes), then applies tSNE and performs MAP-DP clustering on a given number of
components, finally the clustering solutions corresponding to different tSNE projections are combined into a consensus clustering using a meta-clustering algorithm. In this work, we demonstrate the superior performance of our approach to other widely used methods designed to infer the cellular heterogeneity in scRNAseq data.


\poster{A Graph Database Resource for Disease Gene Annotation and Prioritization}{Margherita Mutarelli}{Telethon Institute of Genetics and Medicine, Pozzuoli, italy}
The identification of the molecular cause of genetic diseases is one of the fundamental questions of medical research. Despite the availability of high-throughput techniques like Whole Exome and Whole Genome Sequencing that made now  feasible to genotype patients at an unprecedented level of depth, still a high proportion of cases remain unsolved. We are building a gene and disease network based on the annotated phenotypes present in patients to improve candidate causative gene prioritization.

\poster{Inference on multilayer networks with latent layers}{Alice Tapper}{University of Leeds}
Online social networks often represent one layer in a larger multilayer network of connections between people. Motivated by this, SIS dynamics occurring on two-layer systems with one visible layer and one latent layer are studied. A range of network structures and epidemic parameters are explored using mean field approximations and simulation, with an aim to identify cases where inference is inaccurate if solely the visible layer is analysed.

