\begin{abstract}{Network inference in genomics under censoring}{%
            Veronica Vinciotti}{%
            Brunel University, London, UK}{%
            \KLtag}

Regularized inference of networks using graphical modelling approaches has seen many applications in biology, most notably in the recovery of regulatory networks from high-dimensional gene expression data. Under an assumption of Gaussianity, the popular graphical lasso approach provides an efficient inferential procedure under L1 sparsity constraints. In this talk, I will focus on a latest extension to censored graphical models in order to deal with censored data such as qPCR expression data. We propose a computationally efficient EM-like algorithm for the estimation of the conditional independence graph and thus the recovery of the underlying regulatory network. Similar techniques can be used also in the context of multivariate regression where censored outcomes are to be predicted from a set of predictors. Efficient inferential procedures are presented in the high-dimensional case and pave the way for the development of more complex models that integrate data from different sources and under different mechanisms of missingness.
\end{abstract}
        


\begin{abstract}{Construction of High-resolution Linkage Maps Using Discrete Graphical Models}{%
            Pariya Behrouzi}{%
            Wageningen University, NL }{  \CTtag}
Linkage maps are important for fundamental and applied genetic
research. In this talk, we introduce an algorithm to construct high-quality and
high-density linkage maps for diploid and polyploid species. We employ a sparse
Gaussian copula graphical model and the nonparanormal skeptic approach to
construct linkage maps. We compare our method with other available method
when the data are clean and contain no missing observations and when data are
noisy and incomplete. In addition, we implement the method on real genotype
data of barley and potato. We have implemented the method in the R package "netgwas"
which is freely available at CRAN.
\end{abstract}
        
\begin{abstract}{The Dynamics of Mass and Elite in Dutch Dividend Tax Discourse}{%
            Rafiazka M. Hilman}{%
           Central European University (CEU), Budapest, Hungary}{  \CTtag}
Political sphere in the Netherlands has been passing through torturous way due to the legislation process on Amendment to the Dividend Tax Act 1965 for the last 12 months. Among others, the government proposal on the abolishment of dividend tax (dividendbelasting) becomes the central point. Coalition parties who sponsor this bill, VVD, D66, CDA, and CU, deal with a lot of critiques from opposition in the parliament (Tweede Kamer).
There are three enthralling observations to make in the introduction of this bill. First, it creates ideological distance among central-right coalition parties in which coalition partners D66, CDA, and CU attempt to minimise the political sentiment caused by VVD?s main agenda on the abolishment of dividend tax. On this side, the improvement of investment climate serves as a shield. Secondly, it induces ideological proximity among opposition parties in the parliament where cross-spectrum stands on the same rejection platform by questioning policy cost at 2 billion euros. Last, this political ambiguity leads to diverged public perception related to the importance of public spending over private incentive.
It is the central interest of this research to identify the alignment between public perception and elite discourse captured during the parliamentary debate session. Synthesis and analysis are made in response to two questions: How does social network reflect interactions between political elites, political parties, and mass in dividend tax discourse in the Netherlands? How does the political ambiguity evolve amidst the dynamics of dividend tax discourse?
In order to portray public perception towards elite interaction represented by political key players, social network data are extracted from Twitter in. On top of that minute of meetings recorded during parliament debate session are filtered out to construct the context and sentiment fragmentation among elites. Data are collected using Twitter API service during 4 weeks-period in October 2018. This period is selected to enable the tail-end of dividend tax issue as the government decided to withdraw the plan in the beginning of October 2018. Meanwhile, parliament record is analysed from the initial discussion in November 2017 to October 2018.
The research proceeds as follows: first, the methodology comprising data and model are presented. Next, ideological spectrum and coalition formation becomes a foundation of the following discussion on political dynamics surrounding dividend tax discourse. The final part of the article concludes and discusses the results in terms of network structure and network properties.
\end{abstract}
        



\begin{abstract}{Anomaly detection in networks}{%
            Gesine Reinert}{%
            University of Oxford, UK}{%
            \KLtag}

Detecting financial fraud is a global challenge. This talk will mainly focus on financial transaction networks. In such networks, examples of anomalies are long paths of large transaction amounts, rings of large payments, and cliques of accounts. There are many methods available to detect specific anomalies. Our aim is to detect unknown anomalies. To that purpose we use a strategy with derives features from network comparison methods and spectral analysis, and then apply a random forest method to classify nodes as normal or anomalous. We test the method on synthetic data which we generated, and then on synthetic data without us having had access to the ground truth.
This talk is based on joint work with Andrew Elliott, Mihai Cucuringu, Milton Martinez Luaces, Paul Reidy.

\end{abstract}
        




