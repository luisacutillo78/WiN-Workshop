\begin{abstract}{Data integration using Network and Partial Least Square methods}{%
            Jeanine Houwing-Duistermaat}{%
           School of Mathematics, University of Leeds, Uk}{\KLtag}
The availability of large omics datasets in epidemiological and clinical studies provides many opportunities for research in statistical bioinformatics. The hope is that the abundance of information will provide better understanding of underlying disease mechanisms and accurate prediction models enabling patient targeted screening and treatment. Statistical challenges are to deal with data wrangling, heterogeneity across omic datasets, high dimensionality, data integration and the presence of high correlation within and between datasets (Morris et al, 2017; Houwing-Duistermaat et al, 2017). In this talk I will present Partial Least Squares (PLS) and Network methods for data integration and dimension reduction when analysing several omics datasets simultaneously. The methods will be illustrated by analysis of glycomic datasets and of metabolomics and gene expression in relation with Body Mass Index.
\end{abstract}
        


\begin{abstract}{The micro-dynamic nature of team interactions}{%
            Rebeka O. Szabo}{%
            Central European University (CEU), Budapest, Hungar }{  \CTtag}
Teams have become a popular organization form since well-functioning task-focused groups
are basic pillars of successful organizations. While there is much interest in contemporary social
science in understanding team processes that lead to efficiency, most of these researches rely
heavily on self-reported data yielding static and potentially biased information and tends to
overlook actual interaction processes. We propose a novel approach that allows portraying a
nuanced, complex picture of problem-solving group behaviour by measuring performance
dynamics as it evolves in real-time, in a controlled environment. The research aims to explore
how collaboration networks of small project teams evolve across time and team members, and
how it relates to successful task performance. We investigate interaction patterns in escape
rooms, where all teams are video recorded during the task-solving process in the same
experimental environment. We expected and confirmed that homogeneous distribution of
interaction ties across time and team members fosters successful problem-solving. Concerning
the impact of the initial social roles on the dynamics of the interaction pattern, we hypothesized
that flexible, less hierarchical team structures favour for problem-solving. In the case of the
teams with random composition, the development of a new social structure during the dynamic
performance of an unstructured task is expected to entail more tensions with the conversation
rules than otherwise. This research aims to advance the new science of teams? by focusing on
the network micro-mechanisms that allows us to treat teams as dynamic, adaptive, task-
performing systems.
\end{abstract}
        
        

\begin{abstract}{Bring Mathematics into Biology: Past, Present and Future Impact on Health}{%
            Marta Milo}{%
           University of Sheffield, UK}{%
            \KLtag}
Last decade has seen a massive increase of data production in science. Particularly in the biomedical field, data has grown exponentially thanks to the development of technologies like next generation sequencing and high-throughput quantitative assays. The information that this data contains is only partially uncovered to this date, but the impact that it has on human progression and well-being is already very clear.
Despite the ability to process large amount of data and to quantify fine details of biological processes, the costs, the time to perform such experiments and mainly the complexity of the systems remain in some cases still very prohibitive. 
For this reasons the use of mathematics to study complex systems in its entirety, looking at how they interacts, is having a great impact in current biology and healthcare. 
A variety of statistical, probabilistic and optimisation techniques methods, like machine learning techniques, taht allows to learn from the available data, to detect hidden patterns from large, noisy and complex datasets, is particularly suitable for application in medicine.
In this talk I will present examples of using machine learning techniques for a variety datasets from medical and biological problems and what are the advantages and disadvantages of this approach. 
I will also give examples when tehse techniques enabled to discover infoirmative knowledge from a large complex system in the presence of small number of samples.
Finally I will discuss how we use Machine Learning today for analysis of single-cell sequencing data and how we can use it for future more complex datasets generated integrating data from different sources.
\end{abstract}
% 
       
 \begin{abstract}{Extracting the Value of Big Data}{%
           Denise De Gaetano}{%
          Malta College of Arts Science and Technology}{  \CTtag}
Despite the large volumes of data, companies still struggle to access, manage and extract the information that their day-to-day processes generate. The growth of IT systems has provided these same companies, the ability to capture this potentially valuable data, within a number of applications, databases and organizations. Apart from a strong IT infrastructure, changes in the board and management, within a company also need to be carried out. This allows a number of departments to work in coordination and help ensure success of the utilization of the data at hand. Understanding what is required out of the data, is the first step to generate the best results from the company and customer data. The strategy is to understand how the information can enable an improvement in the business.        
\end{abstract}  
 
\begin{abstract}{Transcriptomic and Genomic Networks}{%
           Arief Gusnanto}{%
          School of Mathematics, University of Leeds}{  \CTtag}
Correlation network is an important tool in bioinformatics to find clusters of genes that are highly correlated in their expressions. The network can be used as a screening method to identify candidate biomarkers and therapeutic targets. This methodology has been successfully implemented in many biological contexts including cancer. While the interpretation may be natural in the context of gene expression data, network of copy number alterations (CNA) is not so straightforward because the data are in segments. CNA are structural variations in the human genome where some regions have more or less copy number than the normal two copies. Since the alterations happen in segments, the data exhibit stronger correlations than gene expression data and the correlations are in ?blocks?. The standard method is no longer adequate for an intuitive interpretation. This talk will describe the research problem, challenges, and our efforts so far in dealing with data from lung cancer patients. This is currently an ongoing joint work with (in alphabetical order) Mohammed Alshahrani, Luisa Cutillo, Charles Taylor, Peter Thwaites, and Henry Wood.
\end{abstract}


 
 
 



