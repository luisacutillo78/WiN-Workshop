\begin{abstract}{An overview on penalized network regression approaches}{%
            Claudia Angelini}{%
           IAC Istituto per le Applicazioni del Calcolo "Mauro Picone", CNR Naples, Italy}{%
            \KLtag}

In this talk we will briefly review the main concepts and problems that arise when analyzing high dimensional data, then we describe recent approaches based on network penalized regression. As an illustrative example, we describe a novel method that combines variable screening and penalized network-based Cox-regression models for the identification of high- and low-risk groups in breast cancer and the selection of potential biomarkers. More in general, we illustrate most recent results and open challenges of network penalized approaches in the context of omic data analysis and integration.
\end{abstract}
        


\begin{abstract}{Construction of High-resolution Linkage Maps Using Discrete Graphical Models}{%
            Marija Mihova}{%
            Ss. Cyril and Methodius University, FINKI, Skopje}{  \CTtag}

Efficient Algorithm for Finding all Minimal Path Vectors in Two-terminal Flow Network
Minimal path and minimal cut vectors are usually used for computing the reliability of a two- terminal flow network with discrete set of possible capacities of its arcs. This work unites the max-flow theory of two-terminal vectors and the theory of minimal path vectors in multi-state systems. Based on obtained theoretical results, we have designed an algorithm for computing all minimal path vectors for a given level d.

\end{abstract}
        
\begin{abstract}{The Dynamics of Mass and Elite in Dutch Dividend Tax Discourse}{%
           Martin Lopez Garcia}{%
           University of Leeds, UK}{  \CTtag}
           On the exact analysis of stochastic epidemic processes on networks
I will show in this talk how to analyse the SIR epidemic model in an exact way when the population under study is formed by a small highly heterogeneous group of N individuals, represented by means of a network. This approach, which amounts to the analysis of the exact $3^N$-states continuous-time Markov chain (CTMC), makes special focus on algorithmic aspects, and requires a creative organization of the space of states ${S,I,R}^N$ of the CTMC. The analysis of the epidemic dynamics is carried out in terms of a number of summary statistics for the disease: (i) the length and size of the outbreak; (ii) the maximum number of individuals simultaneously infected during the outbreak; (iii) the fate of a particular individual within the
population; and (iv) the number of secondary cases caused by a certain individual until she/he recovers. I will illustrate this methodology by studying the spread of the nosocomial pathogen Methicillin-resistant Staphylococcus Aureus among the patients within an intensive care unit (ICU). The interest here is in analysing the effectiveness of different control strategies which intrinsically incorporate heterogeneities among the patients within the ICU.
References:
M. Lopez-Garcia (2016) Stochastic descriptors in an SIR epidemic model for heterogeneous individuals in small networks. Mathematical Biosciences 271: 42-61.
A. Economou, A. Gomez-Corral, M. Lopez-Garcia (2015) A stochastic SIS epidemic model with heterogeneous contacts. Physica A: Statistical Mechanics and its Applications 421: 78-97.
           
\end{abstract}
        

\begin{abstract}{Main challenges in Networks community structure validation}{%
            Luisa Cutillo}{%
            University of Leeds, UK}{%
            \KLtag}
High throughput technologies have led to an increased availability of data and to the need for novel statistical tools. Biological networks provide a mathematical representation of patterns of interaction between appropriate biological elements. We propose a novel approach to compare community structures in different networks. During this seminar we will try to address some open questions: How can we compare two (or more) networks and their community structures? Can we use Network Enrichment Analysis tools to do this? Is it an advantage to integrate metadata to infer communities?
\end{abstract}
 
 
 \begin{abstract}{ROBIN: an R package for validation of community robustness}{%
            Valeria Policastro }{%
            IAC Istituto per le Applicazioni del Calcolo "Mauro Picone", CNR Naples, Italy}{  \CTtag}
In network analysis, many community detection algorithms have been developed. However, their applications leave unaddressed one important question: the statistical validation of the results. We present ROBIN (Robustness In Network), an R package that gives a statistical answer to the validation of the community structure by looking at the robustness of the network. The package implements a methodology presented in a previous paper that detects if the community structure found by a detection algorithm is statistically significant or is a result of chance, merely due to edge positions in the network. The software performs a perturbation strategy and runs a null model to build a set of procedures based on the Variation of Information as a clustering distance. In particular, it provides a procedure to examine the stability of the partition recovered against random perturbations of the original graph structure, a routine to compare different detection algorithms applied to the same network and a graphical interactive representation of networks. The package is useful not only to determine whether the obtained clustering departs significantly from the null model, but also to discover which algorithm better fits for the network of interest.

           
\end{abstract}
        

 
 
 



